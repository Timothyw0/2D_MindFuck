\documentclass[]{article}
\usepackage{datetime}
\usepackage{color,array,graphics}
\usepackage{enumerate}
\usepackage{authblk}
\usepackage{tikz}
\usepackage{titling}
\usepackage{amsthm}
\usepackage{amsmath}
\usepackage{amssymb}
\usepackage[utf8]{inputenc}


\setlength{\textheight}{8.5in}
\setlength{\textwidth}{6.5in}
\setlength{\oddsidemargin}{0in}
\setlength{\evensidemargin}{0in}
\setlength{\droptitle}{-10em}
\voffset0.0in


\title{\textbf{CSCI 4975 Final Project Proposal}}
\author{Pierre Fabris, Timothy Wang}
\date{10/20/17}

\begin{document}
    \maketitle


This language represents a dungeon crawler game maker language.
Different elements, defined the coder, can be placed within the matrix of allocated memory.
Each element of the matrix can store a subroutine, character, and objects that could correspond to different parts of the program.
This program would probably be used to make any kind of custom adventure game within the terminal.
The application does not exclusively have to be centralized toward games.
This program could also be used to make control flow graphs based on descision-making.


\end{document}
