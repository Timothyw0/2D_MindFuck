\documentclass[]{article}
\usepackage{datetime}
\usepackage{color,array,graphics}
\usepackage{enumerate}
\usepackage{authblk}
\usepackage{tikz}
\usepackage{titling}
\usepackage{amsthm}
\usepackage{amsmath}
\usepackage{amssymb}
\usepackage[utf8]{inputenc}


\setlength{\textheight}{8.5in}
\setlength{\textwidth}{6.5in}
\setlength{\oddsidemargin}{0in}
\setlength{\evensidemargin}{0in}
\setlength{\droptitle}{-10em}
\voffset0.0in


\title{\textbf{CSCI 4975 Final Project Proposal}}
\author{Pierre Fabris, Timothy Wang}
\date{10/20/17}

\begin{document}
    \maketitle
Our project 2D MindFuck is based upon the programming language BrainFuck with extensions of the memory
cell representation. In the original BrainFuck language, memory cells are represented linearly whereas 
the user can move the pointer to the left and right to edit the value of the memory cell. This allows 
for a Turing-Complete language that can perform basic arithmetic as well as taking in input and printing
output. For this project, we decided to extend this even further by changing the memory cell representation
into 2 dimensions. This allows for users to traverse the memory cells from left and right as well as up and
down within each memory cell. In order to create this language, we will be using flex and yacc for the 
language scanning and parsing and LLVM will provide the back-end compiling for our language. To go more in depth,
LLVM will be used to manage the memory stored within the memory cells based upon what the user inputs into the 
MindFuck file. 


This language represents a dungeon crawler game maker language.
Different elements, defined the coder, can be placed within the matrix of allocated memory.
Each element of the matrix can store a subroutine, character, and objects that could correspond to different parts of the program.
This program would probably be used to make any kind of custom adventure game within the terminal.
The application does not exclusively have to be centralized toward games.
This program could also be used to make control flow graphs based on descision-making.


\end{document}
